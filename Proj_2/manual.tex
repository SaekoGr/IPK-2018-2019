\documentclass[11pt, a4paper]{article}

\usepackage[left=2cm, text={17cm, 24cm}, top = 3cm]{geometry}
\usepackage{times}
\usepackage[utf8]{inputenc}
\usepackage{verbatim}
\usepackage{amssymb}
\usepackage{multirow}
\usepackage[czech, ruled, noline,  linesnumbered, longend]{algorithm2e}
\usepackage{algpseudocode}
\usepackage[czech]{babel}
\usepackage{graphics}
\usepackage{pdflscape}
\usepackage{hyperref}
\urlstyle{rm}

\begin{document}
\begin{titlepage}
\begin{center}
\Huge
\textsc{Vysoké učení technické v~Brně}\\
\huge
\textsc{Fakulta informačních technologií}\\
\vspace{\stretch{0.382}}
\LARGE 2. projekt z IPK: Scanner sieťových služieb\\
\Huge Dokumentácia
\vspace{\stretch{0.618}}
\end{center}
{\Large 21. apríla 2019 \hfill
Sabína Gregušová}
\end{titlepage}

\tableofcontents
\clearpage

\section{Úvod}
Mojou úlohou bolo vytvoriť scanner sieťových služieb TCP a UDP v jazyku C alebo C++ a na štandartný výstup vypíše v akom stave sa dané porty nachádzajú.

\end{document}