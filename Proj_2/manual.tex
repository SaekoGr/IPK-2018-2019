\documentclass[11pt, a4paper]{article}

\usepackage[left=2cm, text={17cm, 24cm}, top = 3cm]{geometry}
\usepackage{times}
\usepackage[utf8]{inputenc}
\usepackage{verbatim}
\usepackage{amssymb}
\usepackage{multirow}
\usepackage[czech, ruled, noline,  linesnumbered, longend]{algorithm2e}
\usepackage{algpseudocode}
\usepackage[czech]{babel}
\usepackage{graphics}
\usepackage{pdflscape}
\usepackage{hyperref}
\urlstyle{rm}

\begin{document}
\begin{titlepage}
\begin{center}
\Huge
\textsc{Vysoké učení technické v~Brně}\\
\huge
\textsc{Fakulta informačních technologií}\\
\vspace{\stretch{0.382}}
\LARGE 2. projekt z IPK: Scanner sieťových služieb\\
\Huge Dokumentácia
\vspace{\stretch{0.618}}
\end{center}
{\Large 21. apríla 2019 \hfill
Sabína Gregušová}
\end{titlepage}

\tableofcontents
\clearpage

\section{Úvod}
Mojou úlohou bolo vytvoriť scanner sieťových služieb TCP a UDP v jazyku C alebo C++ a na štandartný výstup vypíše v akom stave sa dané porty nachádzajú.

\section{Riešenie}
Rozhodla som sa nepoužiť funkciu na automatické parsovanie argumentov, pretože s ňou nemám dobré skúsenosti a preto som si vytvorila triedu, v ktorej si preparsujem argumenty sama a všetky potrebné informácie si uložím do štruktúry \texttt{Ports}, ktorá obsahuje zoznam portov, flagy, ktoré informujú o existencii viacerých portov v zozname alebo označujú, že zoznam má určitý obsah (oddelený pomlčkou), tak ho rovno peparsuje a uloží. Ďaľšími dôležitými údajmi v štruktúre \texttt{Ports} je doménové meno a IP adresa.

\subsection{Odosielanie paketov}
Najnáročnejšou časťou projektu bolo určite správne odosielanie paketov. Chvíľku mi trvalo uvedomiť si, že je potrebné posielať raw packety, o ktorých ale na internete nie je veľa informácií ktoré by boli použiteľné a aj funkčné. Za najprínosnejší zdroj pre TCP protokol považujem \cite{tcp_tenouk}, pretože sa zameriava práve na problematiku odosielania raw paketov a správnej tvorby IP a TCP hlavičky. Rovnaká stránka veľmi dobre poslúžila aj s článkom o \cite{udp_tenouk} o UDP, kde bolo potrebné zamerať sa na ICMP, keďže inak bez neho by nebolo možné zistiť stav daného portu.

\subsection{TCP skenovanie}


\subsection{UDP skenovanie}

\section{Testovanie}
Kvôli náročnosti projektu mi už nezostal čas na vytvorenie automatických testov. Projekt bol preto testovaný iba ručne 

\section{Záver}
Tento projekt považujem za jeden z najnáročnejších projektov na FIT VUT. So sieťami som pred tým nemala vôbec žiadne skúsenosti a hoci mi učivo prišlo veľmi zaujímavé a podstate zadania som rozumela, bolo veľmi náročné zistiť ako všetko v jazyku C/C++ funguje.

\newpage
\bibliographystyle{czechiso}
\renewcommand{\refname}{Použitá literatúra}
\bibliography{manual}

\end{document}