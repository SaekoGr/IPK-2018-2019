\documentclass[11pt, a4paper]{article}

\usepackage[left=2cm, text={17cm, 24cm}, top = 3cm]{geometry}
\usepackage{times}
\usepackage[utf8]{inputenc}
\usepackage{verbatim}
\usepackage{amssymb}
\usepackage{multirow}
\usepackage[czech, ruled, noline,  linesnumbered, longend]{algorithm2e}
\usepackage{algpseudocode}
\usepackage[czech]{babel}
\usepackage{graphics}
\usepackage{pdflscape}
\usepackage{hyperref}
\urlstyle{rm}

\begin{document}
\begin{titlepage}
\begin{center}
\Huge
\textsc{Vysoké učení technické v~Brně}\\
\huge
\textsc{Fakulta informačních technologií}\\
\vspace{\stretch{0.382}}
\LARGE 1. projekt z IPK: Klient pre OpenWeatherMap API\\
\Huge Dokumentácia
\vspace{\stretch{0.618}}
\end{center}
{\Large 17. marca 2018 \hfill
Sabína Gregušová}
\end{titlepage}

\section{Úvod}

Mojou úlohou bolo vytvoriť klienta pre rozhranie OpenWeatherMap, ktorý bude prostredníctvom dotazov HTTP získavať určité informácie bez použitia knižnice pre spracovávanie protokolu HTTP.
 \section{Riešenie}
Na riešenie tohto projektu som si vybrala jazyk \texttt{Python}. Celkovo som v projekte použila 3 knižnice, a to:
\begin{itemize}
\item \texttt{socket} - vytvorenie socketu a pripojenie k API OpenWeatherMap
\item \texttt{sys} - výpis údajov na \texttt{STDOUT}
\item \texttt{json} - spracovanie dát vo formáte \texttt{JSON}
\end{itemize}

Skript sa spúšta pomocou príkazu \texttt{make run}, a skontroluje si počet vstupných argumentov. Ak je zadaný správny počet argumentov, skript si uloží názov mesta a kľúč pre api, vytvorí \texttt{socket} s flagmi \texttt{AF\_INET} a \texttt{SOCK\_STREAM}. Pomocou funkcie \texttt{connect} sa \texttt{socket} pripojí k rozhraniu OpenWeatherMap a výsledky všetkých týchto príkazov je patrične kontrolované a ošetrené a v prípade chyby sa daný problém vypíše na \texttt{STDERR}.

Po prebratí dát sú dáta parsované pomocou knižnice \texttt{json} a ako prvý je kontrolovaný návratový kód. Návratový kód 200 značí, že dáta boli korektne nájdené. Skript ďalej rozlišuje ešte návratový kódy: 404 - mesto nebolo nájdené alebo neexistuje; 401 - kľúč pre API je nesprávny alebo neplatný, 400 - zlý request a potom ostatné. Pri rozlíšených chybných kódoch je vypísané aj adekvátne hlásenie o danej chybe, pri iných návratových kódoch ale konkrétne informácie už nie sú.

Ak je návratový kód správny, prejde sa k výberu najdôležitejších údajov, konkrétne to je mesto, krajina, počasie, teplota, vlhkosť, tlak, rýchlosť vetra a jeho stupeň. Ak je nejaký údaj nedostupný, tak je jeho hodnota označená ako \texttt{n/a}.

\section{Príklad spustenia}
\textbf{Vstup:}\\
make run api\_key=7d426f17fe39dcf21d359c1a60851baf city=Brno\\

\noindent
\textbf{Výstup:}\\
City: Brno, CZ\\
Weather: Clouds\\
Temperature: 5.93 degrees Celcius\\
Humidity: 72\%\\
Pressure: 1033 hPa\\
Wind speed: 18.0 km/h\\
Wind degree: 300\\

\end{document}